%-------------------------------------------------------------------------------
%	SECTION TITLE
%-------------------------------------------------------------------------------
\cvsection{Projects}


%-------------------------------------------------------------------------------
%	CONTENT
%-------------------------------------------------------------------------------
\begin{cventries}

%---------------------------------------------------------
  \cventry
    {University of Edinburgh} % Role
    {Smart Shelf and Object Retrieval Robot {}} % Event
    {Edinburgh, UK} % Location
    {2020} % Date(s)
    {
      \begin{cvitems} % Description(s)
        \item {A smart shelf that retrieves an object when asked by the user.}
        \item{The object is placed on a smart car which finds the user.}
        \item{The smart car hands the item to the user using a robotic arm}
        \item{On completion, the user can instruct the system to put the object back into the shelf.}
      \end{cvitems}
    }
 
 
  \cventry
    {Personal} % Role
    {Meme Generator {\tiny (\hyperlink{github.com/joanreyero/meme-generator}{github.com/joanreyero/meme-generator})}} % Event
    {} % Location
    {2020} % Date(s)
    {
      \begin{cvitems} % Description(s)
        \item {A web application to generate memes, either randomly from a text and image database, or from user input.}
        \item{Built using \textsc{Python} and Flask, and deployed with Heroku.}
      \end{cvitems}
    }

%---------------------------------------------------------
% \cventry
%     {Personal} % Role
%     {Alchehistory ({\tiny play.google.com/store/apps/details?id=com.SmoketreeStudios.Trees.Alchehistory})} % Event
%     {} % Location
%     {} % Date(s)
%     {
%       A simple Match 3 game revolving around combining elements to form other elements and vice versa. This game was developed in unity3d and featured a selection of small self contained levels and an infinite mode of play.
%     }

\cventry
    {University of Edinburgh}
    {Powergrab {\tiny (github.com/joanreyero/powergrab)}}
    {Edinburgh, UK}
    {2019}
    {
     \begin{cvitems} % Description(s)
        \item {Developed a strategy game in \textsc{Java}, in which a virtual drone flies over the University of Edinburgh campus.}
        \item{The drone has a certain power. Each move consumes power. In the map there are stations: some storing positive power and coins and some negative. The goal is to maximise the coins.}
      \end{cvitems}
    }

\end{cventries}


